\documentclass{article}
\usepackage{amsmath}
\usepackage{ulem}
\usepackage{cancel}
\newcommand{\mbeq}{\overset{!}{=}}


%Für Berechnungen siehe aufgabe16.py

\begin{document}
 In the task on page 16 we were asked to calculate the theoretical values for the experiment on optical resonators. We were given $R=50mm$ for both plano-convex lenses, $L=45mm$ distance between the lenses. The lenses have a width $b=6.35mm$ and a refraction index $n_{L} = 1.515$ The wavelength of the light will be $\lambda = 632nm$ for the following calculations.\\
\\

Upon hitting the first mirror, for the light we have in Matrix representation
\\\\
$M_{boundary 1}= \begin{pmatrix}
1 & 0\\
D_{12}& 1
\end{pmatrix}$ with $D_{12} = -\frac{n_{mirror 1}-n_{air}}{R}$
\\\\
Then for propagation through the first mirror
\\\\
$M_{propagation 1}= \begin{pmatrix}
1 & \frac{b_1}{n_{mirror 1}}\\
0 & 1
\end{pmatrix}$
\\\\
When leaving the mirror
\\\\
$M_{boundary 2}= \begin{pmatrix}
1 & 0\\
D_{23} & 1
\end{pmatrix}$ with $D_{23} = -\frac{n_{air}-n_{mirror 1}}{R'}$
\\\\
Thus in total we have
$M_{mirror 1}= M_{boundary 1} \cdot M_{propagation 1} \cdot M_{boundary 2} 
\\\\
= \begin{pmatrix}
1 & 0\\
D_{12} & 1
\end{pmatrix} \cdot \begin{pmatrix}
1 & \frac{b_1}{n_{mirror 1}}\\
0 & 1
\end{pmatrix} \cdot \begin{pmatrix}
1 & 0\\
D_{23} & 1
\end{pmatrix}= \begin{pmatrix}
\frac{n_2-b \cdot D_{23}}{n_2} & \frac{b}{n_2}\\
-\frac{-D_{12} \cdot n_2 - D_{23}(-D_{12}b+n_2)}{n_2} & \frac{-D_{12}b+n_2}{n_2}
\end{pmatrix}$
\\\\
Since, $n_{mirror1} = n_{mirror 2} \equiv n_2$ and $D_{12} = \frac{n_{mirror 1}-n_{air}}{R}$ and $R \xrightarrow[]{} \infty$ for the first square boundary, $D_{12} \xrightarrow[]{} 0$

$\Rightarrow{} M_{mirror 1}=\begin{pmatrix}
\frac{n_2-b \cdot D_{23}}{n_2} & \frac{b}{n_2}\\
\frac{D_{23}n_2}{n_2} & 1
\end{pmatrix}$
\\
Applying the same method to the second mirror
\\

$M_{mirror 2}= M_{boundary 3} \cdot M_{propagation 2} \cdot M_{boundary 4}$
\\\\
$M_{mirror 2} = \begin{pmatrix}
1 & 0\\
D_{12}' & 1
\end{pmatrix} \cdot \begin{pmatrix}
1 & \frac{b_1}{n_{mirror 1}}\\
0 & 1
\end{pmatrix} \cdot \begin{pmatrix}
1 & 0\\
0 & 1
\end{pmatrix}=\begin{pmatrix}
1 & \frac{b}{n_2}\\
D'_{12} & \frac{D_{23}b+n_2}{n_2}
\end{pmatrix}$
\\
with $ D_{12}' = -\frac{n_{mirror 2}-n_{air}}{R'}$, but this time with a finite $R'$
\\

Therefore, overall
\\\\
$M_{total}= M_{mirror 2} \cdot M'_{prop} \cdot M_{mirror 1}  
\\\\
M_{total}= \begin{pmatrix}
1 & \frac{b}{n_2}\\
D'_{12} & \frac{D_{23}b+n_2}{n_2}
\end{pmatrix}
\cdot
\begin{pmatrix}
1 & 2R'\\
0 & 1
\end{pmatrix} \cdot
\begin{pmatrix}
\frac{n_2-b \cdot D_{23}}{n_2} & \frac{b}{n_2}\\
D_{23} & 1
\end{pmatrix} 
$
\\\\
Now we use $\frac{n-1}{R'} = \frac{1}{f} = 10 m^{-1}, R = 50mm, n=1.515, b=6.35mm $

and calculate
\\\\
$M_{total}= \begin{pmatrix}
2.00 & 0.11\\
30 & 2.08
\end{pmatrix}
$

Using $q = z+iz_R = iz_R$, $z_R = 0.025m$ since $z_R = \frac{\pi \omega^2_0}{\lambda} = \sqrt{\frac{L}{2}(R-\frac{L}{2})}$ since $L=R$ in our confocal arrangement we conclude $z_R=\frac{R}{2}=0.025m$ and thus $q' = \frac{2.00q+0.11}{30q+2.08} = \frac{i2.0(0.025)+0.11}{i30(0.025)+2.08}$
\\\\
Thus:
\\\\
$z'_R = Im(q') = 0.0046692 \approx 4.67\cdot10^{-3}m
\\\\
z' = Re(q') =  0.0536883 \approx 5.37\cdot10^{-2}m
$
\\\\
Hence, we see that the beam has a focus around 5.37 cm after the optical cavity and that the Rayleigh length has decreased considerably, meaning that the beam has further lost collimation.

\end{document}