\documentclass{article}

%Um mit Latex zu arbeiten braucht man zwei Dinge:
% https://miktex.org/download
% https://www.xm1math.net/texmaker/download.html

%Hilfreiche Links:
%Sonderzeichen in Latex einfügen: https://de.wikibooks.org/wiki/LaTeX-Kompendium:_Sonderzeichen

\usepackage{xcolor} %Hier wird ein Paket eingefügt, welches Text farbig machen kann
\usepackage{graphicx}
\usepackage{verbatim}
\usepackage{tabularx}
\usepackage{comment}
\usepackage{mathtools} %Das muss man noch installieren (man wird beim ersten Ausführen des Codes darauf hingewiesen und kann das dann direkt mit 'install' tun)

\DeclarePairedDelimiter\abs{\lvert}{\rvert}%

\begin{document} 

\title{Gaussian beam optics}
\author{Corinna Elena Wegner, Garen Gregorian}
\date{June 8, 2022}
\maketitle %hiermit wird das Deckblatt erst erstellt
\newpage
\tableofcontents
\newpage

\section{Einleitung} %dies ist eine Überschrift
\subsection{Dies ist eine Überschrift zweiten Ranges}
%Wenn ich \subsection* schreibe (mit *), kommt diese Überschrift nicht in die Gliederung

in diesem Versuch soll das Intensitätsprofil von Gaußschen Strahlen vermessen und die Auswirkungen von optischen Elementen untersucht werden\\
%Das muss noch auf Englisch übersetzt werden
%\\ macht einen Zeilenumbruch
%Formeln mit $Formeln$ einfügen
%Wichtigere Formeln mit \[ Formel \]

\textcolor{red}{Genereller Versuchsaufbau mit Skizze}\\
%Dinge die nicht Text sind sondern Dinge die noch rein müssen bzw. TO-DOs hab ich für uns zum Überblick rot gemacht 

\section{Measuring the power of the laser}

In the first part we measured the power of the laser itself. First, we put a photodiode with internal resistance 100k$\Omega$
after the first passive reflector and measured the voltage
\ $U=(1.2 \pm 0.1) V$ %So bindet man Formeln in Text ein
. Then we turned the laser off and measured again at the same position to eliminate the background light (the windows were covered by curtains). We measured 
\ $U_b = (1.2 \pm 0.1)mV$ 
using the 
10k$\Omega$
photodiode. 

\textcolor{red}{add answer, fehlerrechnung, make sure to say we incoroprated the fact that the diode resistances were different and add sktech of experimental set up, answer all questions page 11} 

\paragraph{Calculating the power from the voltage}

When the beam hits the diode, the multimeter detects a voltage $U$. For photodiodes, the relation between the power and light current ist given by 
\begin{equation} 
P = \frac{hfI}{e\eta}
\end{equation}
Here, $h = 6.62607015*10^{-34} Js$ is the planck constant, $e = 1.602176634*10^{-19} C$ the charge of an electron, $I$ is the current and $\eta = 0.75$ the quantum efficiency of the photodiode. We can replace the frequency $f$ in the formula by the corresponding wavelength $\lambda = 632.8 nm$ of the laser using $c= \lambda f$, with the vacuum speed of light  $c = 299792458 \frac{m}{s}$. Using ohm's law $ U= R \cdot I$ we eliminate the current $I$ and obtain 
\begin{equation}
P = \frac{hcU}{\lambda Re \eta}
\end{equation}

\textcolor{red}{Wie Formeln nummerieren? Quellenverzeichnis! https://de.wikipedia.org/wiki/Quantenausbeute}

The resistance $R$ can be read from the photodiode. In the experiment we used two diodes, mainly diode 1 with $R=10 k\Omega$. Diode 2 has $R= 100 k\Omega$.

\textcolor{red}{Fehlerquellen: Lichtstrahl hat Diode nicht perfekt fokussiert, verluste?}

\section{Coupling the optical fibre cables}

After determining the power of the laser we coupled the fibre optic cables into the coupler in the optical path. First, we coupled the multimode cable and adjusted the optical elements such that the conduction was optimized. By variating the angles a little we tried to see different modes on a piece of paper, put behind the cable. Unlike our expectations we could not identify them, instead the dot on the paper disappeared, because too few light was conducted through the cable. We could however see on the paper a dot with a small dark hole in it's center, just like the shape of a donut mode. It is also possible that the small hole was a dust crumb on the lens whatsoever. Unfortunately it was not possible to take a picture of the dot which shows more than a diffuse point, because the mobile phone cameras could not work with the light conditions.\\

Now we measured the voltage from the photodiode behind the cables. Therefor we focused the beam into the photodiode, using another lens. For the multimode cable we measured a maximum voltage of $U=209mV$, but the value fluctuated a lot (about 20mV just from touching the table). After coupling the single mode cable we measured $U=(106 \pm 5)mV$. This is less than for the multimode cable, which makes sense because there is only one mode lead through the single mode cable. With the resistance of the used photodiode, $R = 10k\Omega$, we can calculate the percentage of the power lead through the cable:

\textcolor{red}{Fehlerquellen:
-kabel Beschädigt: keine perfekte Leitung durch kabel, Fehler aus TV1}

\section{Measuring the beam profile}

In this experiment we measured the intensity of the laser light that is emitted by the fibre. We cut off some of the beam with a razor blade to see how much voltage is still measured by the photodiode. Thereby we obtained a profile of the beam cross section. After that we put a lens ($f=100mm$) behind the fibre end so that the beam was focused at the focal point. We measured the profile of the beam at different positions between the two lenses (the second lens ($f=50mm$) focuses the beam into the photodiode). Near the focal point of the first lens, where the waist of the gaussian beam lies, we measured three times. To eliminate influences from ambient light, we normalized the voltage signal with the other photodiode.\\

The total power the photodiode is detecting depends on the position of the razor blade $x$ and is given by:
\begin{equation}
P(x) = \int_x^\infty\mathrm{d}x' I_0 \mathrm{e}^{-\frac{2(x'-x_0)^2}{\omega^2}}.
\end{equation}

\textcolor{red}{Beantworten: Welches Brechungsindexprofil müßte eine Glasfaser aufweisen, damit die Faser eine ideale Gauß-Mode führt?}

\paragraph{Measuring the cross section profile without focusing the beam}

To calculate the beam radius $\omega (z)$ we fitted the data of our measurement a (see annex)
\textcolor{red}{Anhang hinzufügen} %and waist $\omega$ 
to the power integral. In order to do this we calculated the power from the voltage using eq. \\
\textcolor{red}{Formelnummerierung}
\begin{figure}
\includegraphics[width=\textwidth]{part a: cross section profile of collimated beam.png}
\caption{cross section profile of collimated beam}
\label{part_a_fig} %Man kann mit \ref{part_a_fig} auf dieses Bild verweisen
\end{figure}

The first two measurements of the series with $Distance razor blade - fibre end = 8.3 cm$ seemed to fall out of line. When doing the fit, the curve (red dashed) also looked inappropiate. Presumably we measured these points too early, namely not at the point right before the power falls off (i.e. the maximum). Therefore we decided to leave them out of the fitting, which lead to a much better result (see figure ).
\textcolor{red}{which figure number?}
The resulting parameters from the fitting are: \\

For $Distance razor blade - fibre end = 11.0 cm$ the waist is negative, which is impossible and therefore we left this value out in the calculation of the average waist. Since this measurement series was taken at the furthest point from the beam origin, we assume that generally the influence of error sources are much higher than for measurements taken close to the beam origin.\\

\textcolor{red}{include parameters of csv datei a;Interpretation! Ziel war, $\omega$ zu bestimmen.;alle zahlenwerte;messungenauigkeiten;Fehlerquelle: streuendes licht; Am Ende Prüfen ob im Pythoncode die richtigen formeln benutzt wurden, outputs von pythoncode einfügen und diskutieren}

\paragraph{Measuring the cross section profile with lens ($f=100mm$)}

We measured the beam profile using the razor blade technique at seven distances from the lens. Three measurements were taken near the focal point because here we expected the waist $\omega_{0}$, i.e. the minimum of the beam radius $\omega (z)$. They are related by the equation:
\begin{equation}
\omega (z) = \omega_{0}\sqrt{1+\frac{z^2}{z_{R}^2}},
\end{equation}

The razorblade positions z from which we measured the beam profile are $-(7.0\pm 0.2)cm, -(4.0\pm 0.2)cm, -(1.0\pm 0.2)cm, (0.0\pm 0.2)cm, (1.0\pm 0.2)cm, (4.3\pm 0.2) cm$ and $(7.0\pm 0.2)cm$, where $z=0$ is the focal point. For better clearness we plotted the data near focal point in an extra plot. The plots show the calculated powers (eq.) from the measuring data along with the corresponding Fit to the gaussian integral (eq.). From the fit we obtained $I_{0}$ and $\omega(z)$.: \\ 

\textcolor{red}{formelnummerierung und Daten aus pythoncode einfügen}
The value $\omega(7cm) = ... $ is negative and therefore illogical. Therefore we left it out from further calculations. Again this measurement series was taken from the furthest point from the fibre end, so we assume generally higher influence from error sources of any kind\\

We fitted the values $\omega(z)$ to eq. $\omega (z) = \omega_{0}\sqrt{1+\frac{z^2}{z_{R}^2}}$ and eventually obtained the waist $\omega_{0}$ and rayleigh length $z_{R}$ as fit parameters: 
\textcolor{red}{include results} 

where 
\begin{equation}
z_{R} = \frac{\pi\omega_{0}^2 n}{\lambda} 
\end{equation}
 is the rayleigh length, at which the beam radius is $\sqrt{2}\omega_{0}$. In our case, $n=1$ is the refraction index of the medium (air) and the wavelength of the laser is $\lambda = 632.8 nm$.\\

Again, we fitted the measurement series for each razor-lens-distance $z$ to the power integral and obtained the local $\omega(z)$. Then we fitted these together with the corresponding $z$ to eq. []
\textcolor{red}{nummerierung}.
From that we obtained the waist 
\textcolor{red}{$\omega_{0} = ...$ }
The resulting rayleigh length is 
\textcolor{red}{zR ...}

\begin{figure}
\includegraphics[width=\textwidth]{Cross section profile focused beam (near focal point).png} 
\label{near_focal} 
\end{figure}

\includegraphics[width=\textwidth]{Cross section profile focused beam (far from focal point).png}
\includegraphics[width=\textwidth]{beam waist against razor position from focal point.png}

\textcolor{red}{
Wie müssen Sie eine plankonvexe Linse in diesem Fall orientieren, wenn Sie den Einfluß von Linsenfehlern möglichst gering halten wollen?}

\textcolor{red}{
Beschreibung Plot:
Messdaten:
Messungenauigkeiten Ursachen:
-At the focal point the multimeter display showed strong fluctuations}

\section{Optical resonator}

In the last part we constructed an optical resonator and detected the periodic signal of the beam that went through it with an oscilloscope. To do this we focused the beam on a lens 1 ($f=100mm$). Then we focused that on the movable reflector 1 with radius $r = 50mm$. In a distance $L=r$ (confocal arrangement) behind it we then positioned reflector 2 (same radius). The distance between lens 1 and reflector 1 is at first 50mm, so that the focal point of lens 1 lies exactly in the center of the optical resonator. At last we focused the beam into the center of a lens 2 and then into the photodiode with $R=100k\Omega$ (such that the signal is stronger), which we connected to the oscilloscope.

\begin{figure}
\includegraphics[width=\textwidth]{Tv4Aufbau.jpg}
\caption{Experiment 4: Optical resonator}
\label{TV4Aufbau}
\end{figure}


\paragraph{Determining the transmission of the two reflectors}
The mirrors of the Fabry-Perot-interferometer both let some of the incoming beam intensity pass and reflect some. A small amount might also be absorbed by the material of the mirror. The transmitted amplitude $E_{t}$of the interferometer is given by:
\begin{equation}
E_{t} = E_{in}\cdot \frac{T}{1-Re^{i\varphi}}
\end{equation}

Here, $E_{in}$ is the incoming amplitude of the light. The intensity $I_{t}$ of the light is then proportional to $E_{t}^{2}$. In general, the intensity is given by the power divided by the area A ($I=\frac{P}{A}$). The electric power is proportional to the voltage ($P=UI$), where $I$ is the current. In total we obtain the relation:

\begin{equation}
\abs{I_{t}} \propto \abs{E_{t}}^{2} \propto U^{2}
\end{equation}

The transmittivity is then defined as the intensity of the transmitted light divided by the intensity of the incoming light:
\begin{equation}
T = \frac{I_{t}{I_{in}}} = \frac{U_{t}^{2}}{U_{in}^{2}}
\end{equation}

The opposite of the transmissivity $T$ is the reflectivity $R$, if we assume that no light is absorbed by the mirrors they fulfil $R+T=1$. The total transmittivity can be calculated from the transmittivities of the single mirrors by $T = \sqrt{T_{1}T_{2}}$, and analogously  $R = \sqrt{R_{1}R_{2}}$. Having the reflectivity, we can calculate the finesse $F$, a measurement for the quality of the resonator:

\begin{equation}
F= \frac{\pi\sqrt{R}}{1-R}
\end{equation}

We measured the following:\\
$U_b = (29\pm1)mV$ voltage with both mirrors
$U_1 = (55\pm1)mV$  Spiegel voltage with first mirror
$U_2 = (46\pm1)mV$  voltage with second mirror
$U_o = (472\pm1)mV$ voltage without mirrors

From that we obtain 

\begin{comment}
\begin{tabularx}{\textwidth}{p{0.25\textwidth}|r|X|r}
Measurement & ... & Value \\
\hline
\multirow{Transmissivity}
        &Total Transmission & $T= 0.003774956908934214 \pm 2.6083278256172736e-05$\\
        &Transmission of first mirror & $T_{1}= 0.013578174375179546 \pm 2.6083278256172736e-05$\\
        &Transmission of second mirror & $T_{2}= 0.009497989083596666 \pm 2.6083278256172736e-05$\\
        &Calculated total Transmission from $T_{1}$ and $T_{2}$& $T' = 0.01135629129560471 \pm 2.6083278256172736e-05$\\
\hline
\multirow{Reflectivity ($R=1-T$)}
        & R  &$R=  0.9962250430910657 +- 2.6083278256172736e-05$\\
        & R' &$R'=  0.9886437087043953 +- 4.173750866526565e-09$\\
        & Reflectivity of the first mirror &$ R1=  0.9864218256248205 +- 4.970926171216084e-05$\\
        & Reflectivity of the second mirror  &$R2=  0.9905020109164033 +- 4.149125384224495e-05$\\
\hline
\multirow{Finesse}
        & Finesse from total R & $F= 830.6471925693128$\\
        & Finesse from Rp & $Fp = 275.0636820971334$ \\
\end{tabularx} %Warum meckert er da rum und warum ist die Tabelle so hässlich?
\end{comment}

\textcolor{red}{Fehler der Finesse einfügen, Messfehler vielleicht erhöhen, Tabelle fixen} %https://de.wikibooks.org/wiki/LaTeX-Kompendium:_Tabellen

\textcolor{red}{
Pythoncode einbringen
Bestimmen Sie zuerst die Transmission der Spiegel für die vorhandene Wellenlänge. Welche
Reflektivität R und Finesse F sind zu erwarten
Welcher Resonatorkonfiguration entspricht diese Anordnung? Welchen Strahlparameter w0 der Gaußschen Moden erwartenSie für das Lichtfeld im Resonator? Wie groß sollte der Abstand der Linse vom Einkoppelspiegel sein?
Zunächst: Was erwarten Sie für eine Transmissionsfunktion für einen Resonator, der aus Planspiegeln aufgebaut wird und auf den eine monochromatische Lichtwelle trifft? Wie erklären Sie sich das Auftreten von mehr als einem Transmissionsmaximum bei dem gerade aufgebauten Resonator? \\
In einer Periode sollten jetzt nur noch zwei beinahe identische Transmissionsmaxima sichtbar sein. Warum? Drucken Sie das Oszilloskopbild aus. Schätzen Sie das Verhältnis des freien Spektralbereichs zur Linienbreite ab, welche Finesse ergibt sich auf diese Weise?}\\

From the signal detected by the oszilloscope \ref{Osz1} we can also determine the Finesse $F$. The full width at half maximum $\Delta\omega_{FWHM}$ and the free spectral range $\Delta\omega_{FSR}$, which is the distance of two peaks, are related by 
\begin{equation}
\Delta\omega_{FWHM} = \frac{\Delta\omega_{FSR}}{F}
\label{FinesseOszilloskopbild}
\end{equation}
Using the cursors, we see that the full width at half maximum is $\Delta\omega_{FWHM} =3 \pm 1$ units of the oszilloscope pattern. For the free spectral range we see that it is $\Delta\omega_{FSR} = 10 \pm 1$ units wide. Therefore we get a Finesse:\\
$F_{fig} = 3.3333333333333335 \pm 1.11665284679121$

\textcolor{red}{das ist sehr schlecht und weicht stark von der Finesse ab...}

\begin{figure}
\includegraphics[width=\textwidth]{oszilloskopbild.jpg}
\caption{Osz1}
\label{Osz1}
\end{figure}

\begin{comment}
The next figure shows another oscilloscope signal from which we 
\begin{figure}
\includegraphics[width=\textwidth]{oszilloskopbild 2.jpg}
\caption{Osz2}
\label{Osz2}
\end{figure}
\end{comment}

The next figure \ref{Osz3} shows the oscilloscope signal of one single peak:
\begin{figure}
\includegraphics[width=\textwidth]{oszilloskopbild 3.jpg}
\caption{Osz3}
\label{Osz3}
\end{figure}

In the following figure \ref{Eigenmoden_laser} we observe that between the peaks of the modes there are smaller modes which presumably depict the eigenmodes of the laser:\\
\textcolor{red}{Macht das Sinn? Warum sind nicht die Bilder unter den Beschreibungen? und Annex wird komplett ignoriert...}
\begin{figure}
\includegraphics[width=\textwidth]{oszilloskopbild_mit_eigenmoden_vom_laser.jpg}
\caption{Eigenmodenlaser}
\label{Eigenmoden_laser}
\end{figure}

\section{Annex}

\textcolor{red}{Annex noch besser beschriften, Wikipediaquelle}

\subsection{Measurements}

\verbatiminput{Strahlprofil_a.csv}
\verbatiminput{Strahlprofil_b.csv}

\subsection{Pythoncode}

\verbatimimput{matrizenoptik.py} %hier gibts Probleme
\verbatiminput{Strahlprofil_a.py}
\verbatiminput{Strahlprofil_b.py}
\verbatimimput{Optical_resonator.py}

\end{document}